\documentclass[a4paper,11pt]{article}

\usepackage[latin1,utf8]{inputenc}
\usepackage[T1]{fontenc}
\usepackage[french, francais]{babel}
\usepackage{fullpage}
\usepackage{url}

\usepackage{pgf}
\usepackage{tikz}
\usepackage{pgfgantt}
\usepackage{subcaption}
\usepackage{amsmath}
\usepackage{amssymb}
\usepackage{mathrsfs}
\usepackage{xcolor}

\newcommand{\carybe}[1]{ \textcolor{magenta}{#1}}

\title{Rapport projet PIC}

\begin{document}

\maketitle
$$
f_0(x,v,t) = \frac{n_0}{2 \pi v_{th}^2} ( 1 + \alpha cos(k_x x))
 exp( - \frac{v_x^2+v_y^2}{2 v_{th}^2})$$

\section{Solution $\Delta u = f$}
\label{delta u}

On cherche à résoudre l'équation $\Delta u = f$ d'inconnue $u$, avec $\Delta u$ laplacien de u, dans $\mathcal{R}^d$.

$$
\begin{array}{rclr}

\Delta u &=& f \\

\sum\limits_{n=1}^d {\partial^2_ {x_n}} u(x) &=& f(x) \\

\sum\limits_{n=1}^d (ik_n)^2 \widehat{u}(k) &=& \widehat{f}(k) & \text{par transformée de Fourier} \\

-|k|^2  \widehat{u}(k) &=& \widehat{f}(k) & \text{si } \widehat{f}(0)=0 \\

\end{array}
$$

Notons $\mathcal{N} (x) = \sum\limits _{k \in \mathbb{Z}^{d*}} \frac{1}{|k|^2} e^{ik.x}$. En fourier $\widehat{\mathcal{N}}(k) = \frac{1}{|k|^2}$.

D'où :

$$
\begin{array}{rclr}

\widehat{u} (k) &=& - \mathcal{N}(k) \times \widehat{f} (k) \\

u(x) &=& - (\mathcal{N} \ast f) (x)

\end{array}
$$

\section{Application à Vaslov}

L'équation de Vaslov d'inconnue $f$, fonction de $x \in [0, 2 \pi]$, $v$ et $t$ est :
$$
\partial_t f + v \partial_x f - \partial_ x \Phi \partial_v f = 0
$$

Avec : $\Delta \Phi [f](x,t) =  - \int f(x,v,t) dv +  \carybe{\frac{1}{2 \pi}} \int \int f(x, v , t) dx dv$, potentiel d'intéraction.

D'apès la partie \ref{delta u}, $\Phi[f](x,t) = (\mathcal{N} \ast \int f(x,v,t)dv) (t)$, en supposant $\Delta \Phi [f]$ de moyenne nulle.

Soit $(X,V, \beta) = (X_k(t),V_k(t), \beta_k)_{k \in [\![ 1, N]\!]}$ une solution apporché de la fonction de distribution représentant $N$ méta particules de position à l'instant $t$ $X_k(t)$, de vitesse $V_k(t)$ et de masse donnée $\beta_k \in \mathbb{R^+}$. On a pour toutes méta particules $k$ :

$$
\left\lbrace
\begin{array}{rcl}

\frac{dX_k}{dt} &=& V_k \\ \\
\frac{dV_k}{dt} &=& - \partial_x \Phi(X_k, t) \\

\end{array} \right.
$$

Faisons l'hypothèse que $f^n(X,V) = \sum\limits_k \beta_k \delta_{X_k^n} \delta_{ V_k^n}$, avec $f^n(X,V) = f(X,V, t^n)$.

Ainsi :

$$
\begin{array}{rcl}

\Phi[f](x,t^n) &=& (\mathcal{N} \ast \int f(x,v,t^n)dv) (t^n) \\
	&=&  (\mathcal{N} \ast \int \sum\limits_k \beta_k \delta_{X_k^n} \delta_{ V_k^n}dv) (t^n) \\
	&=& (\mathcal{N} \ast \sum\limits_k \beta_k \delta_{X_k^n} \int \delta_{ V_k^n}dv) (t^n) \\
	&=& (\mathcal{N} \ast \sum\limits_k \beta_k \delta_{X_k^n}) (t^n) \\
	&=& \int \mathcal{N}(x-y) \sum\limits_k \beta_k \delta_{X_k^n}(y) dy \\
	&=& \sum\limits_k \beta_k  \int \mathcal{N}(x-y) \delta_{X_k^n}(y) dy \\
	&=& \sum\limits_k \beta_k  \mathcal{N}(x-X_k^n) \\

\end{array}
$$

Et ainsi : 
$$
\boxed{\frac{d V_k} {dt} = -\sum\limits_{l=1}^N \beta_l  \mathcal{N}'(X_k^n-X_l^n)}
$$

En discrétisant le temps, on considère à présent les instants $(t^0, t^1, \dots)$ d'intervalle $\Delta t$. Chaque méta particule est donc représenté par deux suites $(X_k^n, V_k^n)_{n \in \mathbb{N}}$.

$$
\left\lbrace
\begin{array}{rclr}

\frac{X_k^{n+1} - X_k^n}{\Delta t} &=& V_k^{n+1} & (1) \\ \\
\frac{V_k^{n+1} - V_k^n}{\Delta t} &=& - \sum\limits_{l = 1}^N \beta_l  \mathcal{N}'(X_k^n-X_l^n) & (2)\\

\end{array} \right.
$$

L'équation $(1)$ est implicite, pour résoudre ce système, il faut procéder en 2 deux temps :

$$
\phi _ v ^{\Delta t} \left\lbrace
\begin{array}{rclr}

\frac{X_k^{n+1} - X_k^n}{\Delta t} &=& 0  \\ \\
\frac{V_k^{n+1} - V_k^n}{\Delta t} &=& - \sum\limits_{l = 1}^N \beta_l  \mathcal{N}'(X_k^n-X_l^n)\\

\end{array} \right.
$$

$$
\phi _T ^{\Delta t}  \left\lbrace
\begin{array}{rclr}

\frac{X_k^{n+1} - X_k^n}{\Delta t} &=& V_k^{n+1} \\ \\
\frac{V_k^{n+1} - V_k^n}{\Delta t} &=& 0\\

\end{array} \right.
$$

On résoudre en premier le système $\phi _v$ pour calculer $V_k^{n+1}$ puis le système $\phi _t$ pour calculer $X_k^{n+1}$.

$$
(X_k^{n+1}, V_k^{n+1}) = (\phi_T^{\Delta t} \circ \phi_v^{\Delta t}) (X_k^{n}, V_k^{n})
$$

\section{Changement de variables pour obtenir un système Hamiltonien canonique}

On peut résumer ces équations à un système Hamiltonien non cannonnique :

$$
H(X,V) = \frac{1}{2} \sum \limits_{k} \beta_k v_k^2 + \frac{1}{2} \sum \limits _{k, l} \beta_k \beta_l \mathcal{N}(x_k - x_l)
$$

Avec :
$$
\left\lbrace
\begin{array}{rcl}
	\dot{x_k} &=& \frac{1}{\beta_k} \partial_{v_k} H \\
	\dot{v_k} &=& - \frac{1}{\beta_k} \partial_{x_k} H \\

\end{array}
\right.
$$

Pour le rendre canonique posons le changement de variables : 

$$
\left\lbrace
\begin{array}{rcl}
	y_k &=& \sqrt{\beta_k} x_k \\
	z_k &=& \sqrt{\beta_k} v_k \\
\end{array}
\right.
$$

En définissant $\beta$ la matrice diagonale ayant les paramètres $(\beta_i)$ sur sa diagonale  et $\beta_2$ la matrice $(2N,2N)$ diagonale par bloc ayant $\beta$ sur sa diagonale, on peut définir le hamiltonien $K$ :

$$
\beta = \left( \begin{array}{cccc}
					\beta_1 & 0  &\dots & 0 \\
					0 & \beta_2  &\dots & 0 \\
					\vdots  & \vdots  & \ddots &\vdots  \\
					0 & \dots & 0 & \beta_N
\end{array} \right)
$$

$$
\beta_2 = \left( \begin{array}{cccc}
					\beta & 0  \\
					0 & \beta  
\end{array} \right)
$$

$$
\nabla K(Y,Z) = \sqrt{\beta_2^{-1}} \nabla H(X,V) = \left( \begin{array}{c}
	
	\frac{1}{\sqrt{\beta_1}}\partial_{x_1} H(X,V) \\
	\vdots \\
	\frac{1}{\sqrt{\beta_N}}\partial_{x_N} H(X,V) \\
	\frac{1}{\sqrt{\beta_1}}\partial_{v_1} H(X,V) \\
	\vdots \\
	\frac{1}{\sqrt{\beta_N}}\partial_{v_N} H(X,V) 

\end{array} \right)
$$

\carybe{$K(Y,Z) = H(Y,Z)$ a vérifier mais normalement ça marche}

Notons qu'avec la notation $\beta$, on a : $Y = \sqrt{\beta} X$ et $Z = \sqrt{\beta}V$.

Ainsi, on a bien un système hamiltonien canonique
$$\left( \begin{array}{cc}
 Y \\ Z
\end{array} \right) = J \nabla K
$$

avec $J = \left( \begin{array}{cc}
 0 & I \\ -I & 0
\end{array} \right)$

\carybe{notons somme des betas = 1 donc ils sont petits et on veut pas inverser des betas petit}

\carybe{dans le cas ou les betas sont variables (collisons et/ou terme source) ça peut être intéressant}

\end{document}